% Created 2020-09-28 Mon 11:24
% Intended LaTeX compiler: pdflatex
\documentclass[11pt]{article}
\usepackage[utf8]{inputenc}
\usepackage[T1]{fontenc}
\usepackage{graphicx}
\usepackage{grffile}
\usepackage{longtable}
\usepackage{wrapfig}
\usepackage{rotating}
\usepackage[normalem]{ulem}
\usepackage{amsmath}
\usepackage{textcomp}
\usepackage{amssymb}
\usepackage{capt-of}
\usepackage{hyperref}
\usepackage{minted}
\author{Dustin Leatherman}
\date{\today}
\title{Quiz \#2}
\hypersetup{
 pdfauthor={Dustin Leatherman},
 pdftitle={Quiz \#2},
 pdfkeywords={},
 pdfsubject={},
 pdfcreator={Emacs 27.1 (Org mode 9.4)}, 
 pdflang={English}}
\begin{document}

\maketitle

\section{1}
\label{sec:orgb687392}
\begin{quote}
Your daily commute is distributed normally with mean 10 minutes and standard
deviation 2 minutes if there is no convention downtown. However, conventions are
scheduled for roughly 1 in 5 days, and your commute time is distributed normally
with mean 15 minutes and standard deviation 3 minutes if there is a convention.
Let Y be your commute time and \(\theta = 1\) indicate there is a convention and
\(\theta = 0\) if there is no convention.
\end{quote}

The following conditional probabilities are interpreted from the problem description.

\begin{equation}
\begin{split}
f(y | \theta = 0) \sim & N(\mu = 10, \sigma = 2) \label{eq:1}\\
\end{split}
\end{equation}

\begin{equation}
\begin{split}
f(y | \theta = 1) \sim & N(\mu = 15, \sigma = 3) \label{eq:2}\\
\end{split}
\end{equation}

\begin{equation}
\begin{split}
f( \theta = 1) = & \frac{1}{5} \label{eq:3}
\end{split}
\end{equation}

\subsection{a}
\label{sec:org2b1946e}

\begin{quote}
Give an expression for the prior distribution of \(\theta\)
\end{quote}

Since \(\theta\) is either 1 or 0, it can be expressed as a Bernoulli Random
Variable.

\begin{equation}
\begin{split}
f(\theta) = \frac{1}{5}^{\theta} \frac{4}{5}^{1 - \theta}
\end{split}
\end{equation}

\subsection{b}
\label{sec:org50d0a21}

\begin{quote}
Give an expression for the likelihood function. i.e. the PDF of Y given \(\theta\)
\end{quote}

Since \(\theta\) is a Bernoulli random variable and the parameters for
\(\eqref{eq:2}\) assume \(\theta = 1\) whereas \(\eqref{eq:1}\) assumes \(\theta = 0\),
then

\begin{equation}
\begin{split}
f(y | \theta) \sim N (\mu = 10 + 5 \theta, \sigma = \theta + 2) \label{eq:3}
\end{split}
\end{equation}


Thus the likelihood function can be described as the PDF of \(\eqref{eq:3}\)

\begin{equation}
\begin{split}
f(y | \theta) = & \frac{1}{(\theta + 2) \sqrt{2 \pi}} exp(- \frac{(x - (10 + 5 \theta))^2}{2(\theta + 2)^2})\\
= & \frac{1}{(\theta + 2) \sqrt{2 \pi}} exp(- \frac{(x - 10 - 5 \theta)^2}{2(\theta + 2)^2})\\
\end{split}
\end{equation}

\subsection{c}
\label{sec:org788da55}

\begin{quote}
Give an expression for the probability there was a convention downtown given
that your commute time was Y = 16 minutes.
\end{quote}


\begin{equation}
\begin{split}
f(\theta | Y = 16) = & \frac{f(Y = 16 | \theta) f(\theta)}{f(Y = 16 | \theta = 1) \times f(\theta = 1) + f(Y = 16 | \theta = 0) \times f(\theta = 0)}\\
= & \frac{\frac{1}{(\theta + 2) \sqrt{2 \pi}} exp(-\frac{(16 - 10 - 5 \theta)^2}{2 (\theta + 2)^2}) \cdot \frac{1}{5}^{\theta} \frac{4}{5}^{1 - \theta}}{\frac{1}{3 \sqrt{2 \pi}} exp(- \frac{1^2}{2 \times 3^2}) \frac{1}{5} + \frac{1}{2 \sqrt{2 \pi}} exp(- \frac{6^2}{2 \times 2^2}) \frac{4}{5}}\\
= & \frac{1}{(\theta + 2) \sqrt{2 \pi}} exp(-\frac{(6 - 5 \theta)^2}{2 (\theta + 2)^2}) \cdot \frac{1}{5}^{\theta} \frac{4}{5}^{1 - \theta} \cdot \frac{1}{0.02693}\\
\end{split}
\end{equation}


\subsection{d}
\label{sec:org7a161ec}

\begin{quote}
The answer to (c) is a probability assigned to \(\theta\). Given we have observed
\(Y = 16\), is \(\theta\) a random variable? In what sense is the answer to (c) a
meaningful probability? (Answer this from a Bayesian perspective in 50 words or less)
\end{quote}

\(\theta | Y = 16\) is \emph{not} a random variable. If Y is fixed, \(\theta\)
becomes a constant which can be plugged into (c) to produce a probability. The
result of (c) is meaningful because it quantifies the parametric uncertainty for
whether or not a convention is occurring.
\end{document}

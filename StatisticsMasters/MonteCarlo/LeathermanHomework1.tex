% Created 2020-04-05 Sun 07:48
% Intended LaTeX compiler: pdflatex
\documentclass[11pt]{article}
\usepackage[utf8]{inputenc}
\usepackage[T1]{fontenc}
\usepackage{graphicx}
\usepackage{grffile}
\usepackage{longtable}
\usepackage{wrapfig}
\usepackage{rotating}
\usepackage[normalem]{ulem}
\usepackage{amsmath}
\usepackage{textcomp}
\usepackage{amssymb}
\usepackage{capt-of}
\usepackage{hyperref}
\author{Dustin Leatherman}
\date{\today}
\title{Monte Carlo - Homework 1}
\hypersetup{
 pdfauthor={Dustin Leatherman},
 pdftitle={Monte Carlo - Homework 1},
 pdfkeywords={},
 pdfsubject={},
 pdfcreator={Emacs 26.3 (Org mode 9.4)}, 
 pdflang={English}}
\begin{document}

\maketitle
\tableofcontents


\section{1}
\label{sec:org17452d4}

\begin{quote}
The savings account offered by Bank of Bridgeport earns 1.00\% per quarter while
the savings account offered by Lake Michigan Bank earns 0.01\% per day. What are
the annualized rates of return assuming four quarters per year and 365 days per
year for each savings account? Which has a better return?
\end{quote}

\textbf{Bank of Bridgeport}

$$
r_1 = (1 + 0.01)^4 - 1 = 0.04060401 = 4.0604%
$$


\textbf{Lake Michigan Bank}

$$
r_1 = (1 + 0.0001)^365 - 1 = 0.03717241 = 3.7172%
$$

The bank of Bridgeport offers a better return by 0.3432\%.

\section{2}
\label{sec:org0154c41}

\begin{quote}
Consider a lottery with 5 tickets and only one ticket is the winning ticket.
Five people take turns to randomly choose a ticket from the lottery. Assume that
you know little about probability knowledge. How can you use the Monte Carlo
method to estimate the chance of winning for the third person? Briefly describe.
\end{quote}

The probability of choosing a winning ticket is 0.2 and the probability of choosing a
losing ticket is 0.8. Since the probability to win is equal for all tickets, the
arbitrary ticket number W will represent the ``winning'' ticket number. Assuming
that a ticket number is unique, the Monte Carlo method could be applied by
simulating all \(5! = 120\) combinations and the winning percentage for the third
person would be the combinations where W is the third value in an ordered list.
The ratio of combinations where W is the third value and the total combinations.
\end{document}
